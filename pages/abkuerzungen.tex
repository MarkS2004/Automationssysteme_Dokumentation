% Alle Abkürzungen, die in der Arbeit verwendet werden. Die Alphabetische Sortierung übernimmt Latex. Nachfolgend sind Beispiele genannt, welche nach Bedarf angepasst, gelöscht oder ergänzt werden können.
% Die Angaben in der eckigen Klammer werden zur Sortierung der Einträge verwendet. Vor allem bei Formelzeichen hat man sonst das Problem, dass diese möglicherweise nicht wie gewünscht sortiert werden.

% Bei den unten stehenden Formelzeichen ist erläutert, wie explizite Sortierschlüssel über den Inhalt der eckigen Klammer angegeben werden.

% Zum Aktualisieren des Abkürzungsverzeichnisses (Nomenklatur) bitte auf der Kommandozeile folgenden Befehl aufrufen :
% makeindex <Dateiname>.nlo -s nomencl.ist -o <Dateiname>.nls
% Oder besser: Kann in TexStudio unter Tools-Benutzer als Shortlink angelegt werden
% Konfiguration unter: Optionen-Erzeugen-Benutzerbefehle: makeindex -s nomencl.ist -t %.nlg -o %.nls %.nlo

% Allgemeine Abkürzungen %%%%%%%%%%%%%%%%%%%%%%%%%%%%
%\nomenclature[Abb]{Abb.}{Abbildung}
%\nomenclature[bzw]{bzw.}{beziehungsweise}
%\nomenclature[DHBW]{DHBW}{Duale Hochschule Baden-Württemberg}
%\nomenclature[ebd]{ebd.}{ebenda}
%\nomenclaturev[etal]{et al.}{at alii}
\nomenclature[etc]{etc.}{et cetera}
%\nomenclature[evtl]{evtl.}{eventuell}
\nomenclature[f]{f.}{folgende Seite}
\nomenclature[ff]{ff.}{fortfolgende Seiten}
%\nomenclature[ggf]{ggf.}{gegebenenfalls}
%\nomenclature[Hrsg]{Hrsg.}{Herausgeber}
%\nomenclature[Tab]{Tab.}{Tabelle}
%\nomenclature[ua]{u. a.}{unter anderem}
%\nomenclature[usw]{usw.}{und so weiter}
\nomenclature[vgl]{vgl.}{vergleiche}
\nomenclature[zB]{z. B.}{zum Beispiel}
%\nomenclaturev[zT]{z. T.}{zum Teil}

% Dateiendungen %%%%%%%%%%%%%%%%%%%%%%%%%%%%%%%%%%%%
\nomenclature[EMF]{EMF}{Enhanced Metafile}
\nomenclature[JPG]{JPG}{Joint Photographic Experts Group}
\nomenclature[PDF]{PDF}{Portable Document Format}
\nomenclature[PNG]{PNG}{Portable Network Graphics}
%\nomenclature[]{XML}{Extensible Markup Language}

% Abkürzungen von Fachbegriffen %%%%%%%%%%%%%%%%%%%%
\nomenclature[ABS]{ABS}{Antiblockiersystem}
\nomenclature[ESC]{ESC}{Electronic Stability Control, Fahrdynamikregelung}

% Formelzeichen %%%%%%%%%%%%%%%%%%%%%%%%%%%%%%%%%%%%
\nomenclature[a]{$a$}{Beschleunigung}
\nomenclature[F]{$F$}{Kraft}
\nomenclature[m]{$m$}{Masse}
\nomenclature[P]{$P$}{Leistung}
\nomenclature[U]{$U$}{Spannung}
\nomenclature[R]{$R$}{Widerstand}


