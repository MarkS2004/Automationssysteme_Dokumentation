% Alle Abkürzungen, die in der Arbeit verwendet werden. Die Alphabetische Sortierung übernimmt Latex. Nachfolgend sind Beispiele genannt, welche nach Bedarf angepasst, gelöscht oder ergänzt werden können.

% Allgemeine Abkürzungen %%%%%%%%%%%%%%%%%%%%%%%%%%%%%%%%%%%%
% Aus Sicht des Autors dieser Vorlage müssen allgemein bekannte Abkürzungen nich erläutert werden. Dieses Beispiel ist nur deshalb gezeigt, um die Technik zu demonstrieren, wie man mit \glsunset die Abkürzung im Text verwenden kann, aber kein Eintrag im Abkürzungsverzeichnis erfolgt (siehe Abschnitt 2 in Kapitel 2 Grundlagen).
\newacronym{acr:zb}{z.~B.}{zum Beispiel}

% Dateiendungen %%%%%%%%%%%%%%%%%%%%%%%%%%%%%%%%%%%%
\newacronym{acr:EMF}{EMF}{Enhanced Metafile}
\newacronym{acr:JPG}{JPG}{Joint Photographic Experts Group}
\newacronym{acr:KI}{KI}{Künstliche Intelligenz}
\newacronym{acr:PDF}{PDF}{Portable Document Format}
\newacronym{acr:PNG}{PNG}{Portable Network Graphics}

% Abkürzungen von Fachbegriffen %%%%%%%%%%%%%%%%%%%%
\newacronym{acr:ABS}{ABS}{Antiblockiersystem}
\newacronym{acr:ESC}{ESC}{Electronic Stability Control}





